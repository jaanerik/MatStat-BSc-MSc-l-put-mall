\section*{Sissejuhatus}
\addcontentsline{toc}{section}{Sissejuhatus}  

Lõputöö algab alati sissejuhatusega, mille täpse sõnastamise võib jätta töö kirjutamise lõpufaasi.

Sissejuhatuses tutvustatakse uuritavat probleemi ja hüpoteese ning vajadusel selgitatakse tööga seotuid mõisteid.\footnote{Definitsioonid tuuakse välja sissejuhatuse lõpuosas. Lisaks, sissejuhatuses ei tasu teemast liialt kõrvale kalduda -- hea sissejuhatus on konkreetne ning siin ei tasu minna liiga isiklikuks.}
Sissejuhatuses võib välja tuua ka selle, miks valitud teemat on tarvis lähemalt uurida, kas teema on aktuaalne ning kas töö omab ka praktilist väljundit.
Sissejuhatuses võib kirjeldada ka andmeid ja nende päritolu ning selgitada töö metoodikat.
Sissejuhatuse lõpuosas kirjeldatakse tavaliselt töö struktuuri ning tuuakse välja, mida peatükkides käsitletakse.

Siin jagatud soovitused ei ole kivisse raiutud -- alati juhindu oma juhendaja soovitustest (isegi kui see on vastuolus siinse juhendiga).
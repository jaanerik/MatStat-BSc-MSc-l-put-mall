\section*{Kokkuvõte}
\addcontentsline{toc}{section}{Kokkuvõte}  

Lõputöö viimane peatükk on alati kokkuvõte.
Kokkuvõttes antakse vastused sissejuhatuses esitatud küsimustele ning kirjeldatakse lühidalt töö metoodikat ja tulemusi.
Hea kokkuvõte on lühike ning ülevaatlik.
Kokkuvõttes võib kirjeldada ka lõputöö nõrku kohti ja puudujääke (eriti selliseid, mis võivad töö tulemuste usaldatavust vähendada), kui neist pole töö varasemas osas kirjutatud.
Tihtipeale kirjeldatakse kokkuvõttes ka töö võimalikku edasist jätku, näiteks mõne sarnase teema uurimist.

Kui kokkuvõte tuleb liiga pikk, võib tulemuste tõlgendustest teha eraldi peatüki (enne kokkuvõtet), mis on tavaliselt pealkirjaga "`Arutelu"'.

Kokkuvõttele järgneb kasutatud materjalide loetelu.
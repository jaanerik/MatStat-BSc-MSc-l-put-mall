\subsubsection*{Lihtlitsents lõputöö reprodutseerimiseks ja üldsusele kättesaadavaks tegemiseks}

Mina, \textcolor{red}{autor},

\begin{enumerate}
\item annan Tartu Ülikoolile tasuta loa (lihtlitsentsi) minu loodud teose \textcolor{red}{lõputöö pealkiri}, mille juhendaja on \textcolor{red}{juhendaja nimi}, reprodutseerimiseks eesmärgiga seda säilitada, sealhulgas lisada digitaalarhiivi DSpace kuni autoriõiguse kehtivuse lõppemiseni.

\item  Annan Tartu Ülikoolile loa teha punktis 1 nimetatud teos üldsusele kättesaadavaks Tartu Ülikooli veebikeskkonna, sealhulgas digitaalarhiivi DSpace kaudu Creative Commonsi litsentsiga CC BY NC ND 3.0, mis lubab autorile viidates teost reprodutseerida, levitada ja üldsusele suunata ning keelab luua tuletatud teost ja kasutada teost ärieesmärgil, kuni autoriõiguse kehtivuse lõppemiseni.

\item  Olen teadlik, et punktides 1 ja 2 nimetatud õigused jäävad alles ka autorile.

\item Kinnitan, et lihtlitsentsi andmisega ei riku ma teiste isikute intellektuaalomandi ega isikuandmete kaitse õigusaktidest tulenevaid õigusi.
\end{enumerate}

\textcolor{red}{Autor}\\
\textcolor{red}{kuupäev}\\


\subsubsection*{Kommentaarid (ära seda osa töösse sisse jäta)}
Käesolev litsents on tavaline lihtlitsents lõputöö elektroonseks avaldamiseks (40), kuid on võimalik, et on vaja hoopis \href{http://dok.ut.ee/wd/?page=pub_list_dynobj&desktop=57835&tid=70993&data_only=true&search=Otsi&field_100193_search_type=ANY&field_100193_text_search_value=ppimine}{lihtlitsentsi 42 või 44}.
Juhuks, kui on vajadus kinnise kaitsmise või lõputöö avaldamisele piirangute kehtestamiseks, siis vajaliku taotluse malli leiab number \href{http://dok.ut.ee/wd/?page=pub_list_dynobj&desktop=57835&tid=70993&data_only=true&search=Otsi&field_100193_search_type=ANY&field_100193_text_search_value=ppimine}{39} alt.

Lihtlitsentsi pole tarvis allkirjastada.

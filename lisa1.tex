\section*{Lisa 1. RxInfer kood lihtsa mudeli defineerimiseks} 
\label{section:lisa1}
\addcontentsline{toc}{section}{Lisa 1. RxInfer kood lihtsa mudeli defineerimiseks} 

\begin{minted}{julia}
using RxInfer
using Distributions
using GraphViz

struct EmissionNode{T <: Real} <: DiscreteUnivariateDistribution
    x :: T
end

@node EmissionNode Stochastic [out, x]
@rule EmissionNode(:out, Marginalisation) (q_x :: Bernoulli,) = PointMass(y)
@rule EmissionNode(:x, Marginalisation) (q_out :: PointMass,) = begin
    tmpx0 = pdf(Normal(0,1), q_out.point)
    tmpx1 = pdf(Normal(0,1), q_out.point-1)
    p = tmpx1/(tmpx0+tmpx1)
    return Bernoulli(p)
end
@marginalrule EmissionNode(:out_x) (q_out::PointMass, q_x::Bernoulli) = begin
    tmpn0 = pdf(Normal(0,1), q_out.point)
    tmpn1 = pdf(Normal(0,1), q_out.point-1)
    tmpp1 = q_x.p
    tmpp0 = 1-tmpp1  
    Z = tmpn0*tmpp0 + tmpn1 * tmpp1
    p = tmpn1 * tmpp1 / Z
    return (out = q_out, x = Bernoulli(p))
end

@average_energy EmissionNode (q_out::PointMass, q_x::Bernoulli) = begin
    -1/2 * log(2 * π) - q_x.p/2 * (q_out.point-1)^2 - (1-q_x.p)/2*q_out.point^2
end

@model function my_model(y)
    x ~ Bernoulli(0.75)
    y ~ EmissionNode(x)
end
a = infer(
    model = my_model(),
    data = (y = 1.0,),
    options = (limit_stack_depth = 500,),
    free_energy = true,
    iterations = 2,    
)
\end{minted}